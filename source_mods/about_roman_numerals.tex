\section[Appendix: Roman Numerals]{Roman Numerals}
\label{sec:about=roman_numerals}

Traditional Roman numerals are made up of the following denominations:

\begin{center}
I = 1\\
V = 5\\
X = 10\\
L = 50\\
C = 100\\
D = 500\\
M = 1000
\end{center}

In order for a number written in Roman numerals to be considered valid
there are three basic rules which must be followed.

\begin{center}
\textbf{Numerals must be arranged in descending order of size.}\\
\textbf{{M}, {C}, and {X} cannot be equalled or exceeded by smaller
denominations.}\\
\textbf{{D}, {L}, and {V} can each only appear once.}
\end{center}

For example, the number sixteen could be written as {XVI} or {XIIIIII},
with the first being the preferred form as it uses the least number of
numerals. We could not write {IIIIIIIIIIIIIIII} because we are making
{X} (ten) from smaller denominations, nor could we write {VVVI} because
the second and third rule are being broken.

The "descending size" rule was introduced to allow the use of
subtractive combinations. For example, four can be written {IV} because
it is one before five. As the rule requires that the numerals be
arranged in order of size it should be clear to a reader that the
presence of a smaller numeral out of place, so to speak, was
unambiguously to be subtracted from the following numeral rather than
added.

For example, nineteen could be written {XIX} = {X} (ten) + {IX} (nine).
Note also how the rule requires {X} (ten) be placed before {IX} (nine),
and {IXX} would not be an acceptable configuration (descending size
rule). Similarly, {XVIV} would be invalid because {V} can only appear
once in a number.

Generally the Romans tried to use as few numerals as possible when
displaying numbers. For this reason, {XIX} would be the preferred form
of nineteen over other valid combinations, like {XIIIIIIIII} or
{XVIIII}.

By mediaeval times it had become standard practice to avoid more than
three consecutive identical numerals by taking advantage of the more
compact subtractive combinations. That is, {IV} would be written instead
of {IIII}, {IX} would be used instead of {IIIIIIIII} or {VIIII}, and so
on.

In addition to the three rules given above, if subtractive combinations
are used then the following four rules must be followed.

\begin{enumerate}
\def\labelenumi{\roman{enumi}.}
\tightlist
\item
  Only one {I}, {X}, and {C} can be used as the leading numeral in part
  of a subtractive pair.
\item
  {I} can only be placed before {V} and {X}.
\item
  {X} can only be placed before {L} and {C}.
\item
  {C} can only be placed before {D} and {M}.
\end{enumerate}

Which means that {IL} would be considered to be an invalid way of
writing forty-nine, and whereas {XXXXIIIIIIIII}, {XXXXVIIII}, {XXXXIX},
{XLIIIIIIIII}, {XLVIIII}, and {XLIX} are all quite legitimate, the
latter is the preferred (minimal) form. However, minimal form was not a
rule and there still remain in Rome many examples where economy of
numerals has not been employed. For example, in the famous Colosseum the
numerals above the forty-ninth entrance is written {XXXXVIIII} rather
than {XLIX}.

It is also expected, but not required, that higher denominations should
be used whenever possible; for example, {V} should be used in place of
{IIIII}, {L} should be used in place of {XXXXX}, and {D} should be used
in place of {CCCCC}. However, in the church of Sant'Agnese fuori le Mura
(St Agnes' outside the walls), found in Rome, the date, {MCCCCCCVI}
(1606), is written on the gilded and coffered wooden ceiling; I am sure
that many would argue that it should have been written {MDCVI}.

So if we believe the adage, "when in Rome do as the Romans do," and we
see how the Romans write numerals, then it clearly gives us much more
freedom than many would care to admit.