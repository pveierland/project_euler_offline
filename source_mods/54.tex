\section[Problem \#54: Poker hands]{Poker hands}
\label{sec:problem_54}

In the card game poker, a hand consists of five cards and are ranked,
from lowest to highest, in the following way:

\begin{itemize}
\tightlist
\item
  \textbf{High Card}: Highest value card.
\item
  \textbf{One Pair}: Two cards of the same value.
\item
  \textbf{Two Pairs}: Two different pairs.
\item
  \textbf{Three of a Kind}: Three cards of the same value.
\item
  \textbf{Straight}: All cards are consecutive values.
\item
  \textbf{Flush}: All cards of the same suit.
\item
  \textbf{Full House}: Three of a kind and a pair.
\item
  \textbf{Four of a Kind}: Four cards of the same value.
\item
  \textbf{Straight Flush}: All cards are consecutive values of same
  suit.
\item
  \textbf{Royal Flush}: Ten, Jack, Queen, King, Ace, in same suit.
\end{itemize}

The cards are valued in the order:\\
2, 3, 4, 5, 6, 7, 8, 9, 10, Jack, Queen, King, Ace.

If two players have the same ranked hands then the rank made up of the
highest value wins; for example, a pair of eights beats a pair of fives
(see example 1 below). But if two ranks tie, for example, both players
have a pair of queens, then highest cards in each hand are compared (see
example 4 below); if the highest cards tie then the next highest cards
are compared, and so on.

Consider the following five hands dealt to two players:

\begin{longtable}[]{@{}lllllll@{}}
\toprule
\textbf{Hand} & ~ & \textbf{Player 1} & ~ & \textbf{Player 2} & ~ &
\textbf{Winner} \\
\endhead
\midrule
\textbf{1} & ~ & \begin{minipage}[t]{0.2\linewidth}\raggedright
5H 5C 6S 7S KD\\

{\small{}Pair of Fives}\strut
\end{minipage} & ~ & \begin{minipage}[t]{0.2\linewidth}\raggedright
2C 3S 8S 8D TD\\

{\small{}Pair of Eights}\strut
\end{minipage} & ~ & Player 2 \\
\textbf{2} & ~ & \begin{minipage}[t]{0.2\linewidth}\raggedright
5D 8C 9S JS AC\\

{\small{}Highest card Ace}\strut
\end{minipage} & ~ & \begin{minipage}[t]{0.2\linewidth}\raggedright
2C 5C 7D 8S QH\\

{\small{}Highest card Queen}\strut
\end{minipage} & ~ & Player 1 \\
\textbf{3} & ~ & \begin{minipage}[t]{0.2\linewidth}\raggedright
2D 9C AS AH AC\\

{\small{}Three Aces}\strut
\end{minipage} & ~ & \begin{minipage}[t]{0.2\linewidth}\raggedright
3D 6D 7D TD QD\\

{\small{}Flush with Diamonds}\strut
\end{minipage} & ~ & Player 2 \\
\textbf{4} & ~ & \begin{minipage}[t]{0.2\linewidth}\raggedright
4D 6S 9H QH QC\\

{\small{}Pair of Queens\\
Highest card Nine}\strut
\end{minipage} & ~ & \begin{minipage}[t]{0.2\linewidth}\raggedright
3D 6D 7H QD QS\\

{\small{}Pair of Queens\\
Highest card Seven}\strut
\end{minipage} & ~ & Player 1 \\
\textbf{5} & ~ & \begin{minipage}[t]{0.2\linewidth}\raggedright
2H 2D 4C 4D 4S\\

{\small{}Full House\\
With Three Fours}\strut
\end{minipage} & ~ & \begin{minipage}[t]{0.2\linewidth}\raggedright
3C 3D 3S 9S 9D\\

{\small{}Full House\\
with Three Threes}\strut
\end{minipage} & ~ & Player 1 \\
\bottomrule
\end{longtable}

The file, \href{project/resources/p054_poker.txt}{poker.txt}, contains
one-thousand random hands dealt to two players. Each line of the file
contains ten cards (separated by a single space): the first five are
Player 1's cards and the last five are Player 2's cards. You can assume
that all hands are valid (no invalid characters or repeated cards), each
player's hand is in no specific order, and in each hand there is a clear
winner.

How many hands does Player 1 win?