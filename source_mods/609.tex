\section[Problem \#609: $π$ sequences]{$π$ sequences}
\label{sec:problem_609}

For every $n \ge 1$ the \textbf{prime-counting} function
$\pi(n)$ is equal to the number of primes not exceeding
$n$.\\
E.g. $\pi(6)=3$ and $\pi(100)=25$.

We say that a sequence of integers $u  = (u_0,\cdots,u_m)$ is a
\emph{$\pi$ sequence} if

\begin{itemize}
\tightlist
\item
  $u_n \ge 1$ for every $n$
\item
  $u_{n+1}= \pi(u_n)$
\item
  $u$ has two or more elements
\end{itemize}

For $u_0=10$ there are three distinct $\pi$
sequences: (10,4), (10,4,2) and (10,4,2,1).

Let $c(u)$ be the number of elements of
$u$ that are not prime.\\
Let $p(n,k)$ be the number of $\pi$
sequences $u$ for which $u_0\le n$ and
$c(u)=k$.\\
Let $P(n)$ be the product of all $p(n,k)$
that are larger than 0.\\
You are given: P(10)=3×8×9×3=648 and P(100)=31038676032.

Find $P(10^8)$. Give your answer modulo 1000000007.