\section[Problem \#122: Efficient exponentiation]{Efficient exponentiation}
\label{sec:problem_122}

The most naive way of computing \emph{n}\textsuperscript{15} requires
fourteen multiplications:

\begin{center}
\emph{n} × \emph{n} × ... × \emph{n} = \emph{n}\textsuperscript{15}
\end{center}

But using a "binary" method you can compute it in six multiplications:

\begin{center}
\emph{n} × \emph{n} = \emph{n}\textsuperscript{2}\\
\emph{n}\textsuperscript{2} × \emph{n}\textsuperscript{2} =
\emph{n}\textsuperscript{4}\\
\emph{n}\textsuperscript{4} × \emph{n}\textsuperscript{4} =
\emph{n}\textsuperscript{8}\\
\emph{n}\textsuperscript{8} × \emph{n}\textsuperscript{4} =
\emph{n}\textsuperscript{12}\\
\emph{n}\textsuperscript{12} × \emph{n}\textsuperscript{2} =
\emph{n}\textsuperscript{14}\\
\emph{n}\textsuperscript{14} × \emph{n} = \emph{n}\textsuperscript{15}
\end{center}

However it is yet possible to compute it in only five multiplications:

\begin{center}
\emph{n} × \emph{n} = \emph{n}\textsuperscript{2}\\
\emph{n}\textsuperscript{2} × \emph{n} = \emph{n}\textsuperscript{3}\\
\emph{n}\textsuperscript{3} × \emph{n}\textsuperscript{3} =
\emph{n}\textsuperscript{6}\\
\emph{n}\textsuperscript{6} × \emph{n}\textsuperscript{6} =
\emph{n}\textsuperscript{12}\\
\emph{n}\textsuperscript{12} × \emph{n}\textsuperscript{3} =
\emph{n}\textsuperscript{15}
\end{center}

We shall define m(\emph{k}) to be the minimum number of multiplications
to compute \emph{n}\textsuperscript{\emph{k}}; for example m(15) = 5.

For 1 ≤ \emph{k} ≤ 200, find {∑} m(\emph{k}).