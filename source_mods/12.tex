\section[Problem \#12: Highly divisible triangular number]{Highly divisible triangular number}
\label{sec:problem_12}

The sequence of triangle numbers is generated by adding the natural
numbers. So the 7\textsuperscript{th} triangle number would be 1 + 2 + 3
+ 4 + 5 + 6 + 7 = 28. The first ten terms would be:

\begin{center}1, 3, 6, 10, 15, 21, 28, 36, 45, 55, ...\end{center}

Let us list the factors of the first seven triangle numbers:

\begin{longtable}[]{@{}rl@{}}
\textbf{1}: & 1\\
\textbf{3}: & 1, 3\\
\textbf{6}: & 1, 2, 3, 6\\
\textbf{10}: & 1, 2, 5, 10\\
\textbf{15}: & 1, 3, 5, 15\\
\textbf{21}: & 1, 3, 7, 21\\
\textbf{28}: & 1, 2, 4, 7, 14, 28\\
\end{longtable}

We can see that 28 is the first triangle number to have over five
divisors.

What is the value of the first triangle number to have over five hundred
divisors?