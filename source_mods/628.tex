\section[Problem \#628: Open chess positions]{Open chess positions}
\label{sec:problem_628}

A position in chess is an (orientated) arrangement of chess pieces
placed on a chessboard of given size. In the following, we consider all
positions in which $n$ pawns are placed on a
$n \times n$ board in such a way, that there is a single pawn in
every row and every column.

We call such a position an \emph{open} position, if a rook, starting at
the (empty) lower left corner and using only moves towards the right or
upwards, can reach the upper right corner without moving onto any field
occupied by a pawn.

Let $f(n)$ be the number of open positions for a
$n \times n$ chessboard.\\
For example, $f(3)=2$, illustrated by the two open positions
for a $3  \times 3$ chessboard below.

\begin{longtable}[]{@{}lll@{}}
\includegraphics{project/images/p628_chess4.png} & ~ &
\includegraphics{project/images/p628_chess5.png} \\
\end{longtable}

You are also given $f(5)=70$.

Find $f(10^8)$ modulo $1\,008\,691\,207$.