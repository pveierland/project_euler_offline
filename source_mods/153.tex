\section[Problem \#153: Investigating Gaussian Integers]{Investigating Gaussian Integers}
\label{sec:problem_153}

As we all know the equation \texttt{x}\textsuperscript{2}=-1 has no
solutions for real \texttt{x}.\\
If we however introduce the imaginary number \texttt{i} this equation
has two solutions: \texttt{x=i} and \texttt{x=-i}.\\
If we go a step further the equation
(\texttt{x}-3)\textsuperscript{2}=-4 has two complex solutions:
\texttt{x}=3+2\texttt{i} and \texttt{x}=3-2\texttt{i}.\\
\texttt{x}=3+2\texttt{i} and \texttt{x}=3-2\texttt{i} are called each
others' complex conjugate.\\
Numbers of the form \texttt{a}+\texttt{bi} are called complex numbers.\\
In general \texttt{a}+\texttt{bi} and \texttt{a}−\texttt{bi} are each
other's complex conjugate.

A Gaussian Integer is a complex number \texttt{a}+\texttt{bi} such that
both \texttt{a} and \texttt{b} are integers.\\
The regular integers are also Gaussian integers (with \texttt{b}=0).\\
To distinguish them from Gaussian integers with \texttt{b} ≠ 0 we call
such integers "rational integers."\\
A Gaussian integer is called a divisor of a rational integer \texttt{n}
if the result is also a Gaussian integer.\\
If for example we divide 5 by 1+2\texttt{i} we can simplify
$\dfrac{5}{1 + 2i}$ in the following manner:\\
Multiply numerator and denominator by the complex conjugate of
1+2\texttt{i}: 1−2\texttt{i}.\\
The result is $\dfrac{5}{1 + 2i} = \dfrac{5}{1 + 2i}\dfrac{1 - 2i}{1 - 2i} = \dfrac{5(1 - 2i)}{1 - (2i)^2} = \dfrac{5(1 - 2i)}{1 - (-4)} = \dfrac{5(1 - 2i)}{5} = 1 - 2i$.\\
So 1+2\texttt{i} is a divisor of 5.\\
Note that 1+\texttt{i} is not a divisor of 5 because
$\dfrac{5}{1 + i} = \dfrac{5}{2} - \dfrac{5}{2}i$.\\
Note also that if the Gaussian Integer (\texttt{a}+\texttt{bi}) is a
divisor of a rational integer \texttt{n}, then its complex conjugate
(\texttt{a}−\texttt{bi}) is also a divisor of \texttt{n}.

In fact, 5 has six divisors such that the real part is positive: \{1, 1
+ 2\texttt{i}, 1 − 2\texttt{i}, 2 + \texttt{i}, 2 − \texttt{i}, 5\}.\\
The following is a table of all of the divisors for the first five
positive rational integers:

\begin{longtable}[]{@{}lll@{}}
\toprule
\texttt{n} & \begin{minipage}[t]{0.3\linewidth}\raggedright
Gaussian integer divisors\\
with positive real part\strut
\end{minipage} & \begin{minipage}[t]{0.3\linewidth}\raggedright
Sum s(\texttt{n}) of\\
these divisors\strut
\end{minipage} \\
\endhead
\midrule
1 & 1 & 1 \\
2 & 1, 1+\texttt{i}, 1-\texttt{i}, 2 & 5 \\
3 & 1, 3 & 4 \\
4 & 1, 1+\texttt{i}, 1-\texttt{i}, 2, 2+2\texttt{i}, 2-2\texttt{i},4 &
13 \\
5 & 1, 1+2\texttt{i}, 1-2\texttt{i}, 2+\texttt{i}, 2-\texttt{i}, 5 &
12 \\
\bottomrule
\end{longtable}

For divisors with positive real parts, then, we have:
$\sum \limits_{n = 1}^{5} {s(n)} = 35$.

For $\sum \limits_{n = 1}^{10^5} {s(n)} = 17924657155$.

What is $\sum \limits_{n = 1}^{10^8} {s(n)}$?